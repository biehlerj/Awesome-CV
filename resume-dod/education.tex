%-------------------------------------------------------------------------------
%	SECTION TITLE
%-------------------------------------------------------------------------------
\cvsection{Education}


%-------------------------------------------------------------------------------
%	CONTENT
%-------------------------------------------------------------------------------
\begin{cventries}

	\cventry
	{Security+}
	{CompTIA}
	{N/A}
	{Jan. 10, 2020}
	{
		\begin{cvitems} % Description(s) bullet points
			\item {Successfully passed renewal test January 10th, 2022.}
			\item {Completed renewal course.}
			\item {Certification expires January 10th, 2026}
		\end{cvitems}
	}

	\cventry
	{Associate's of Science in Business Administration}
	{San Diego Mesa College}
	{San Diego, CA, USA}
	{Aug. 2012 - Jun. 2017}
	{}

	%---------------------------------------------------------
\end{cventries}
